\documentclass[a4paper,12pt]{article} 
\usepackage[T2A]{fontenc}			
\usepackage[utf8]{inputenc}			
\usepackage[english,russian]{babel}	
\usepackage{amsmath,amsfonts,amssymb,amsthm,mathtools} 
\usepackage[colorlinks, linkcolor = blue]{hyperref}
\usepackage{upgreek}\usepackage[left=2cm,right=2cm,top=2cm,bottom=3cm,bindingoffset=0cm]{geometry}
\usepackage{multirow}
\usepackage{graphicx}
\usepackage{xcolor}
\usepackage{multirow}

\author{Шелихов Дмитрий\\Группа Б01-305}

\title{\textbf{Работа 3.1.3\\Измерение магнитного поля Земли}} 
\date{\today}

\begin{document} 

\maketitle

\textbf{Цель работы:} исследовать свойства постоянных неодимовых магнитов; измерить с их помощью горизонтальную и вертикальную составляющие индукции магнитного поля Земли и магнитное наклонение.

\textbf{В работе используются:} неодимовые магниты; тонкая нить для изготовления крутильного маятника; медная проволока; электронные весы; секундомер; измеритель магнитной индукции; штангенциркуль; брусок; линейка и штатив из немагнитных материалов; набор гирь и разновесов.

\noindent\textbf{Теоретическая справка}

$$ \vec{m} = I\vec{S} $$ (1) - магнитный момент тонкого витка с током. 
$$ \vec {B_{дип}} = \frac {\mu_0}{4\pi}(\frac {3(\vec {m} \cdot \vec{r})\vec{r}}{r^5} - \frac {\vec{m}}{r^3})  $$ (2) - Магнитное поле точечного диполя. 
$$ \vec{M} = [\vec{m} \times \vec{B}]  $$ (3) - Механический момент сил, действующий на точечный магнитный диполь $\vec{m}$ 
$$ W = -(\vec{m} \cdot \vec{B})  $$ (4) - Потенциальная энергия, которой обладает диполь с постоянным $\vec{m}$
$$ \vec{F} = (\vec{m} \cdot \nabla)\vec{B}  $$ (5) - Сила, действующая на магнитный диполь в неоднородном внешнем поле $\vec{B}$.
$$ F_{12} = -\frac {6m_1m_2}{r^4}  $$ (6.1) - Сила, взаимодействия двух точечных диполей, когда их моменты направлены вдоль соединяющей их прямой.
$$ F_{12} = \frac {3m_1m_2}{r^4}  $$ (6.2) - Сила, взаимодействия, если моменты направлены перпендикулярно соединяющей их прямой.

\noindent\textbf{Экспериментальная установка}

Используем неодимовые шарообразные магниты: 
	а) Вещество магнитожёстко
	б) Шары намагничены однородно

$$ \vec {B_0} = \frac {\mu_0 \vec{m}}{2\pi R^3} $$ (7) - магнитное поле внутри шара (однородно).
$$ \vec {m} = \vec{M}V $$ (8), где \vec{M} - намагниченность материала магнита.
$$ \vec {B_r} = 4\pi\vec{M}  $$ (9) - остаточная индукция материала. 
$$ B_p = B_0 = \frac {2}{3}B_r  $$ (10) - индукция на полюсах однородно намагниченного шара. 

\noindent\textbf{Определение магнитного момента магнитных шариков}

\textbf{Метод А.} 



\end{document}
